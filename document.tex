\documentclass[a4paper]{article}
\usepackage{amsmath}
\usepackage{amssymb}
\usepackage{graphicx}
\usepackage{verbatim}
\usepackage[bookmarks=true]{hyperref}
\usepackage[left=2cm, right=2cm]{geometry}
\usepackage{listings}
\usepackage[dvipsnames]{xcolor}

\definecolor{wow}{RGB}{73,63,72}

\lstset{morecomment=[l]--,
		commentstyle=\color{NavyBlue},
		numbers=left,
        classoffset=0,
		morekeywords={TODO},
		keywordstyle=\color{red}\bfseries,
		classoffset=1,
		morekeywords={class,
					  visible,
					  temporal,
                      TD,
                      TI, 
                      items,
					  modules,
                      connection,
                      predicates,
                      axioms,
                      end,
                      inherits,
                      vars},
		keywordstyle=\color{wow}\bfseries,
        classoffset=2,
        morekeywords={axioms},keywordstyle=\color{green},
        classoffset=0}%



\title{Politecnico di Milano\\
Formal Methods for Concurrent \\
and \\
Real-time Systems\\
Homework project\\
\textbf{Collaborative Robotics Modeling }}
\author{Aldeghi Gabriele \\
  Mantovani Mirko \and
  Sacco Alessio \\
  Sonzogni Stefano}
 \date{\today}
 



% The other logical operators are: \lnot, \land, \lor, \iff
\newcommand{\liff}{\iff}
\DeclareMathOperator{\limply}{\Rightarrow}
\DeclareMathOperator{\future}{Future}
\DeclareMathOperator{\past}{Past}
\DeclareMathOperator{\lastsOp}{Lasts}
\newcommand{\lasts}{\lastsOp_{ee}}
\newcommand{\lastsie}{\lastsOp_{ie}}
\newcommand{\lastsii}{\lastsOp_{ii}}
\newcommand{\lastsei}{\lastsOp_{ei}}
\DeclareMathOperator{\lastedOp}{Lasted}
\newcommand{\lasted}{\lastedOp_{ei}}
\DeclareMathOperator{\always}{Always}
\DeclareMathOperator{\sometimeFuture}{SometimeFut}
\DeclareMathOperator{\sometimePast}{SometimePast}
\DeclareMathOperator{\withinOp}{Within}
\newcommand{\within}{\withinOp_{ee}}
\DeclareMathOperator{\uptonow}{UpToNow}
\DeclareMathOperator{\becomes}{Becomes}
\DeclareMathOperator{\nexttime}{NextTime}
\DeclareMathOperator{\lasttime}{LastTime}
\DeclareMathOperator{\until}{Until}
\DeclareMathOperator{\since}{Since}


\begin{document}
\maketitle
\begin{center}
    \includegraphics[width=7cm]{images/polimi-logo}
\end{center}
\clearpage
{\hypersetup{hidelinks}\tableofcontents}
\clearpage

\section{Formalization of the problem}
\subsection{Problem description}
brief description

scheme with areas subdivision
\subsection{Definitions and Acronyms of Components}
\begin{itemize}
    \item \textbf{R}: The whole KUKA mobile robot
    \item \textbf{EE}:\@ The End-Effector of the robot's arm
    \item \textbf{O}:\@ The Operator that works in the same environment of the Robot and interacts with it
    \item \textbf{L}:\@ The entire Layout in which Robot and Operator work
    \item \textbf{BA}:\@ The Bin Area (top-right)
    \item \textbf{WP}:\@ The single WorkPiece which is transported by the robot
    \item \textbf{HID}:\@ The Human Interface Device used by the Operator to control the Robot
\end{itemize}

\subsection{Constants}
\begin{itemize}
    \item \textbf{N}:\@ The capacity of the local bin of the Robot
\end{itemize}
\subsection{World discretization}

\subsubsection{Human body parts}
\begin{itemize}
    \item \textbf{Head area}:\@ Highly sensitive areas
    \item \textbf{Arms area}:\@ Very delicate areas
    \item \textbf{Body area}:\@ Delicate areas
\end{itemize}

\subsubsection{Robot parts}
\begin{itemize}
    \item \textbf{Head area}:\@ Arm
    \item \textbf{Arms area}:\@ Cart
\end{itemize}

\subsubsection{Robot speed}
\begin{itemize}
    \item \textbf{None}:\@ Null speed, the robot cannot move
    \item \textbf{Low}:\@ Low speed, the robot can move every other time instant
    \item \textbf{Normal}:\@ Normal speed, the robot can move at every time instant
\end{itemize}

% \subsubsection{Layout areas}
% We discretized the layout in such a way that the most important and critical areas, in which Human-Robot interaction are very likely to happen, have a fine-grained grid, whereas the other areas, in which the Operator should not be present to assist the Robot, are modeled as bigger blocks in order not to introduce unnecessary complexity in the model and to avoid state space explosion.
% \begin{figure}[htp] 
% \includegraphics[width=\textwidth]{images/layout} 
% \caption{Subdivision of the layout, highlighted areas are the most dangerous ones} 
% \label{fig:layout} 
% \end{figure}

% \clearpage
% The most critical areas are the ones with indices from 1 to 12, particular attention must be paid to L\textsubscript0 too, since it is where the EE and O could be working together. 

% \begin{figure}[htp] 
% \includegraphics[width=\textwidth]{images/layoutnames} 
% \caption{Names of the splitted layout and various areas} 
% \label{fig:layout2} 
% \end{figure}


\subsubsection{Layout areas}
We discretized the layout in such a way that the most important and critical areas, in which Human-Robot interaction are very likely to happen, have a fine-grained grid, whereas the other areas, in which the Operator should not be present to assist the Robot, are modeled as bigger blocks in order not to introduce unnecessary complexity in the model and to avoid state space explosion.

The most critical areas are the ones with indices from 1 to 12, particular attention must be paid to L\textsubscript0 too, since it is where the EE and O could be working together. 

\begin{figure}[htp] 
\includegraphics[width=0.9\textwidth]{images/layoutnames} 
\caption{Subdivision of the layout. The highligthed areas are the dangerous ones.} 
\label{fig:layout2} 
\end{figure}

\subsection{Assumptions and modeling}
\paragraph{Robot}
The robot is discretized in two parts, the cart and the arm. The cart is such that it can only occupy one area at the time: this is a strong assumption, however it is vital because it reduces the complexity of our model. The arm is modeled by considering two joints and the end effector. The first joint is assumed to be the connection between the cart and the arm, therefore its position is at all times the same as the cart position. The second joint allows broader movement to the arm and it links the first joint to the end effector. The whole arm can reach at most a distance of two adjacent areas. The response of the actuators to a change in the control variables is assumed to be short enough to be modeled as instantaneous.

\paragraph{Human}
The human is discretized in body, head and arms. Each one of the parts can only occupy one area. The movement can only be observed, but not controlled, by the robot.

\clearpage
\section{Archi-TRIO model}




\clearpage
\section{TRIO+ specification}

\subsection{Definition of Grid class}
\begin{lstlisting}[fontadjust, mathescape, frame=single] 
-- This class is the one that provides the predicates for adjacency
-- and the different typologies of position that we can have.
-- We have three typologies:
--     - Border:  They are all the areas near the walls of the room or near the
--                working positions
--     - Danger:  They are all the areas in which the operator can work. 
--                The robot need to be extra careful in these areas, as
--                through its arm it can severely harm the operator
--     - Transit: They are all the remaining areas that are not Border or
--                Danger areas.
-- 
-- All the trio classes that needs to control the adjacency between 
-- different position will import this module.

class Grid
    visible adjacent;

    temporal domain integer;

    TI items
        predicates 
            adjacent(position, position);
                    
            adjToBinArea({L0, L1, L2, L3, L4, L5, L6, L7, L8, L9,
                          L10, L11, L12, L13, L14, L15, L16, L17,
                          L18, L19, L20, L21, L22, L23, L24, L25,
                          L26, L27, L28, LBA, LCB}),
                          
            adjToTombstone({L0, L1, L2, L3, L4, L5, L6, L7, L8, L9,
                            L10, L11, L12, L13, L14, L15, L16, L17,
                            L18, L19, L20, L21, L22, L23, L24, L25,
                            L26, L27, L28, LBA, LCB}),
                            
            adjToConveyorBelt({L0, L1, L2, L3, L4, L5, L6, L7, L8,
                               L9, L10, L11, L12, L13, L14, L15, 
                               L16, L17, L18, L19, L20, L21, L22,
                               L23, L24, L25, L26, L27, L28, LBA,
                               LCB}),
                               
            border({L0, L1, L2, L3, L4, L5, L6, L7, L8, L9,
                    L10, L11, L12, L13, L14, L15, L16, L17,
                    L18, L19, L20, L21, L22, L23, L24, L25,
                    L26, L27, L28, LBA, LCB}),
             
                    
            danger({L0, L1, L2, L3, L4, L5, L6, L7, L8, L9,
                    L10, L11, L12, L13, L14, L15, L16, L17,
                    L18, L19, L20, L21, L22, L23, L24, L25,
                    L26, L27, L28, LBA, LCB}),
                    
            transit({L0, L1, L2, L3, L4, L5, L6, L7, L8, L9,
                    L10, L11, L12, L13, L14, L15, L16, L17,
                    L18, L19, L20, L21, L22, L23, L24, L25,
                    L26, L27, L28, LBA, LCB});

    axioms
        
        -- This is the formula that appears into the document's appendix.
        -- Specifies which areas are adjacent one with another
        adjacency:
           $\forall x, y (adjacent(x,y) \limply ... );$
        
        -- Definition of all the areas that are danger and which are not.
        dangerArea:
           $danger(L0) \land danger(L2) \land danger(L3) \land danger(L4) \land danger(L6) \land danger(L7) \land$
           $danger(L8) \land danger(L10) \land danger(L11) \land danger(L12) \land danger(L15) \land danger(L17) \land$
           $danger(L20) \land danger(L22) \land danger(L23) \land danger(L24) \land$
          $\neg danger(L1) \land \neg danger(L5) \land \neg danger(L9) \land \neg danger(L13) \land \neg danger(L14) \land$
          $\neg danger(L16) \land \neg danger(L18) \land \neg danger(L19) \land \neg danger(L21) \land \neg danger(L25) \land$
          $\neg danger(L26) \land \neg danger(L27) \land \neg danger(L28);$
        
        -- Definition of all the border areas.
        borderArea:
           $border(L1) \land border(L2) \land border(L3) \land border(L4) \land border(L25) \land border(L27) \land$
           $border(L28) \land border(L24) \land border(L23) \land border(L22) \land border(L20) \land border(L17) \land$
           $border(L15) \land border(L14) \land border(L13) \land border(L18) \land border(L9) \land border(L5) \land $
          $\neg border(L0) \land \neg border(L6) \land \neg border(L7) \land \neg border(L8) \land \neg border(L10) \land$
          $\neg border(L11) \land \neg border(L12) \land \neg border(L21) \land \neg border(L16) \land \neg border(L19);$

        -- Definition of all the transit areas.
        transitArea:
            $\forall x (transit(x) \iff \neg border(x) \lor \neg danger(x));$

        binArea:
            $\forall x (adjToBinArea(x) \iff adjacent(x, LBA);$

        tombstone:
            $\forall x (adjToTombstone(x) \iff adjacent(x, L0);$

        conveyorBelt:
            $\forall x (adjToConveyorBelt(x) \iff adjacent(x, LCB);$
end Grid.\end{lstlisting}

\pagebreak
\subsection{Definition of the PositionSensor class}
\begin{lstlisting}[fontadjust, mathescape, frame=single]
    
-- This is the basic sensor of our system. Allows to define a predicate position,
-- that will be used by the robot's position sensors.   
class PositionSensor
    visible position, adjacent;

    temporal domain integer;

    -- The domain of the predicate position lies inside all the possible areas
    -- of interest. 
    -- The predicate moved allows us to manage the robot's movement. This is due
    -- to the fact that the robot can move at three speed intensity. Therefore, 
    -- for a correct modelization of this movement, we needed a predicate that 
    -- tell us whether the robot has moved in this time instant.
    TD items 
        predicates  position({L0, L1, L2, L3, L4, L5, L6, L7,
                              L8, L9, L10, L11, L12, L13, L14,
                              L15, L16, L17, L18, L19, L20, L21,
                              L22, L23, L24, L25, L26, L27, L28,
                              LBA, LCB});
                    moved();

    modules: Grid: Grid;

    connection: {(Grid.adjacent, adjacent)};

    axioms
        moved:$ moved() \iff \exists x \past(position(x), 1) \land \neg position(x);$

end PositionSensor.\end{lstlisting}

\pagebreak
\subsection{Definition of the operator's sensor classes}
\begin{lstlisting}[fontadjust, mathescape, frame=single] 
class OperatorArmPositionSensor
        inherits PositionSensor

        visible position;

        temporal domain integer;

        TD items
  
  axioms
            -- unique position
            uniqueArms:$ \exists x (position(x) \land \neg\exists y (x != y \land position(y)));$
end OperatorArmPositionSensor.\end{lstlisting}
\begin{lstlisting}[fontadjust, mathescape, frame=single] 
class OperatorBodyPositionSensor
    inherits PositionSensor

    visible position;

    temporal domain integer;

    TD items
        axioms
            -- the body is only in one area
            uniqueBody:$ \forall x (position(x) \limply \neg \exists y (position(y) \land x != y));$

end OperatorBodyPositionSensor.\end{lstlisting}
\pagebreak
\begin{lstlisting}[fontadjust, mathescape, frame=single] 
class OperatorHeadPositionSensor
    inherits PositionSensor
    
    visible position;

    temporal domain integer;

    axioms
        -- The head can be only in one area at each time
        headIsUnique:$ \exists x (position(x) \land \forall y (y != x \limply \neg position(y)));$

end OperatorHeadPositionSensor.\end{lstlisting}
\begin{lstlisting}[fontadjust, mathescape, frame=single] 
class OperatorPositionSensor
    inherits PositionSensor

    visible arms, body, head;
    
    temporal domain integer;

    TD items
        predicates
            arms({L0, L1, L2, L3, L4, L5, L6, L7, L8, L9,
                  L10, L11, L12, L13, L14, L15, L16, L17, L18, L19,
                  L20, L21, L22, L23, L24, L25, L26, L27, L28 }),
            body({L0, L1, L2, L3, L4, L5, L6, L7, L8, L9,
                  L10, L11, L12, L13, L14, L15, L16, L17, L18, L19,
                  L20, L21, L22, L23, L24, L25, L26, L27, L28 }),
            head({L0, L1, L2, L3, L4, L5, L6, L7, L8, L9,
                  L10, L11, L12, L13, L14, L15, L16, L17, L18, L19,
                  L20, L21, L22, L23, L24, L25, L26, L27, L28 });


    modules LeftArm: OperatorArmPositionSensor,
             RightArm: OperatorArmPositionSensor,
             Body: OperatorBodyPositionSensor,
             Head: OperatorHeadPositionSensor,
             Grid: Grid;

    axioms
        -- connect the predicates between the modules
        arms:$ \forall x (arms(x) \iff (LeftArm.position(x) \lor RightArm.position(x)));$
        body:$ \forall x (body(x) \iff Body.position(x));$
        head:$ \forall x (head(x) \iff Head.position(x));$

        -- head is on the body or in a close by cell
        headOnTheBody:$ \forall x (head(x) \limply body(x) \lor \exists y (body(y) \land Grid.adjacent(x, y)));$

        -- arms are on the body or in a close by cell
        armsOnTheBody:$ \forall x (arms(x) \limply body(x) \lor \exists y (body(y) \land Grid.adjacent(x, y)));$

        -- the operator can move one area in each time instant
        movement:$ \forall x (body(x) \limply \exists y (\future(body(y), 1) \land (x == y \lor Grid.adjacent(x, y))));$

end OperatorPositionSensor.
\end{lstlisting}

\pagebreak
\subsection{Definition of the robot's sensor classes}
\begin{lstlisting}[fontadjust, mathescape, frame=single] 
-- This class specify the sensors for the robot's arm. We decide
-- to model the arm as the real one present on the KUKA unit.
-- The arm is composed of 3 links and one end effector.
-- There are three predicates that will be use to determine
-- when the robot is switching between an action an another,
-- and they are pickedUp(), holding() and dropped().

class RobotArmPositionSensor
    inherits PositionSensor

    temporal domain integer;

    visible position, link3, link2, endEffector, pickedUp, holding, dropped;

    modules Grid: Grid;

    TD items
        predicates
            -- Link3 is the first segment of the arm from the body of the robot
            link3({ L0, L1, L2, L3, L4, L5, L6, L7, L8, L9,
                    L10, L11, L12, L13, L14, L15, L16, L17,
                    L18, L19, L20, L21, L22, L23, L24, L25,
                    L26, L27, L28}),
                    
            -- Link2 is the second segment of the arm, linked to link3 and link1
            link2({ L0, L1, L2, L3, L4, L5, L6, L7, L8, L9,
                    L10, L11, L12, L13, L14, L15, L16, L17,
                    L18, L19, L20, L21, L22, L23, L24, L25,
                    L26, L27, L28}),
                    
            -- Link1 is the third segment of the arm, which is linked to link2 
            -- and holds the end effector
            link1({ L0, L1, L2, L3, L4, L5, L6, L7, L8, L9,
                    L10, L11, L12, L13, L14, L15, L16, L17,
                    L18, L19, L20, L21, L22, L23, L24, L25,
                    L26, L27, L28}),
                    
            TODO
            TODO The endeffector should be able to go over areas L0, LBA, LCB
            TODO         
            -- End effector is the "hand" of the robot
            endEffector({ L0, L1, L2, L3, L4, L5, L6, L7, L8, L9,
                          L10, L11, L12, L13, L14, L15, L16, L17,
                          L18, L19, L20, L21, L22, L23, L24, L25,
                          L26, L27, L28}),
                    

            -- Signal if the pick action has been completed successfully
            pickedUp(),

            -- Signal to model the fact that the robot is holding a piece
            holding(),

            -- Signal to model the fact that the robot has dropped the piece it
            -- was holding
            dropped();

            -- Signal to model the fact that the robot's arm is being touched by
            -- the operator
            contact();

    axioms:

        -- The position is all the cells that are occupied by the arm
        armPosition: 
            $\forall x (position(x) \iff link2(x) \lor link1(x));$
        
        -- link 2 is connected to link3
        connection32: 
            $\forall x \forall y (link2(x) \implies$
                 $link3(x) \lor (x \neq y \land link3(y) \land Grid.adjacent(x, y)));$

        -- link1 is connected to link2
        connection21: 
            $\forall x \forall y (link1(x) \implies$ 
                 $link2(x) \lor (x \neq y \land link2(y) \land Grid.adjacent(x, y)));$

        -- end effector is on link1
        endEffectorOnLink1:
            $\forall x (endEffector(x) \iff link1(x));$

        -- link2 is unique
        uniqueLink2: 
            $\forall x \forall y (x \neq y \land link2(x) \implies \neg link2(y));$

        -- link1 is unique
        uniqueLink1: 
            $\forall x \forall y (x \neq y \land link1(x) \implies \neg link1(y));$

        -- end effector occupies only one position at a time
        onlyOneEndEffector:
            $\forall x (endEffector(x) \iff \nexists y (x \neq y \land endEffector(y)));$

            


        -- Only one of pickedUp(), holding() and dropped() can be true in 
        -- each instant
        onlyOne:
            $(pickedUp() \implies$
                $\neg holding() \land \neg dropped()) \land$
            $(holding() \implies$ 
                $\neg pickedUp() \land \neg dropped()) \land$
            $(dropped() \implies$
                $\neg pickedUp() \land \neg holding());$

        -- holding() holds between pickedUp() and dropped()
        sequence:
            $\becomes(\neg pickedUp()) \land \until(holding(), dropped());$

end RobotArmPositionSensor.
\end{lstlisting}
\begin{lstlisting}[fontadjust, mathescape, frame=single] 
-- The robot has a position, and from that position it can be present only in
-- Grid.adjacent areas from the position. The Position predicate represent 
-- these adjacent position, while the position predicate specify the current 
-- central position of the robot.
-- Another condition we need to supply is the fact that the robot cannot 
-- "teleport". This condition is guaranteed by the fact that any new position
-- that the robot assumes must be Grid.adjacent to at least one position of 
-- the robot in the past.

class RobotCartPositionSensor; 
    inherits PositionSensor
    
    visible position, moved;
    
    temporal domain integer;

    modules Grid: Grid;
    
    axioms
    
        -- the cart always exists
        existsCart: $\exists x (position(x));$

        -- the cart is only in one area
        uniqueCart: $\forall x (position(x) \implies \nexists y (position(y) \land x \neq y));$         

        -- An area occupied by the robot must be Grid.adjacent with an area
        -- occupied by the robot one time instant in the past
        doNotTeleport:
            $\forall x (\becomes(position(x)) \implies 
                \exists y (Past(position(y), 1) \land Grid.adjacent(x, y)));$
            
end RobotCartPositionSensor.
\end{lstlisting}
\pagebreak
\begin{lstlisting}[fontadjust, mathescape, frame=single] 
-- This is the class that collects all the sensor of the
-- robot.
-- In this class we defined a predicate for the robot's velocities,
-- both for the arm and the cart. We defined axioms to check the
-- robot's movement. As example, if the robot is moving at low speed,
-- it can move in a adjacent area in two time unit. This modelization
-- is permitted through the use of a support predicate moved().
-- We also define a predicate contact to model the fact that the robot's
-- arm can be touched by the operator to guide its working process.
-- Some specific axioms has been defined to model the situation
-- explained before.

class RobotPositionSensor
    temporal domain integer

    visible Cart, Arm;

    modules Cart: RobotCartPositionSensor,
             Arm: RobotArmPositionSensor,
             Status: RobotStatus,
             Grid: Grid;

    axioms
        -- link3 is in the same position as the cart
        link3OnCart: 
            $\forall x (Arm.link3(x) \iff Body.position(x));$

        -- cartTargetSpeed defines how often the robot cart can move in the grid
        cartSpeedNone: 
            $Status.currentCartSpeed == None \implies$ 
                $\exists x \future(Cart.position(x), 1) \land Cart.position(x);$

                
        cartSpeedLowMoved: 
            $\exists x (Status.currentCartSpeed == Low \land Cart.moved() \implies$
                $Cart.position(x) \land \future(Cart.position(x), 1));$

        cartSpeedLowNotMoved: 
            $\exists x (Status.currentCartSpeed == Low \land$
              $\neg Cart.moved() \implies$
                  $Cart.position(x) \land$ 
                  $\exists y (\future(Cart.position(y), 1) \land$ 
                    $x != y \land$ 
                    $Grid.adjacent(x, y)));$

        cartSpeedNormal: 
            $\exists x (Status.currentCartSpeed == Normal \land$
              $Cart.position(x) \implies$
                  $\exists y (\future(Cart.position(y), 1) \land$
                    $x != y \land$
                    $Grid.adjacent(x, y)));$

        -- if the speed is low, then the arm can move from one area to another
        -- only half of the time
        armSpeedLowMoved: 
            $\exists x (Status.currentCartSpeed == None \land$
              $Status.currentArmSpeed == Low \land Arm.moved() \implies$ 
                  $Arm.endEffector(x) \land \future(Arm.endEffector(x), 1));$

        armSpeedLowNotMoved: 
            $\exists x (Status.currentCartSpeed == None \land$
              $Status.currentArmSpeed == Low \land$ 
              $\past(\neg Arm.moved() \land$
              $Arm.endEffector(x), 1) \implies$ 
                  $Arm.endEffector(y) \land$
                  $x != y \land$ 
                  $Grid.adjacent(x, y)));$

        -- if the speed is normal, then the arm can move from one area to 
        -- another at every time instant
        armSpeedNormal: 
            $\exists x (Status.currentCartSpeed == None \land$
              $Status.currentArmSpeed == Normal \land$
              $Arm.endEffector(x) \implies$
                  $\exists y (\future(Arm.endEffector(y), 1) \land$
                    $x != y \land$
                    $Grid.adjacent(x, y)));$

        -- if the cart is moving, then the arm must stay on top of the cart,
        -- both the end effector and the link
        armOnTopMovingCart: 
            $\forall x (Status.currentCartSpeed != None \land$
              $Cart.moved() \land$ 
              $Cart.position(x) \implies$
                  $Arm.position(x));$









        --------------------
        -- CONTACT AXIOMS --
        --------------------

        -- the Arm.contact() predicate can be true only when the arms of the 
        -- operator are in the same area as the arm of the robot
        armContact:
            $\forall x (Arm.position(x) \land Arm.contact() \implies$
                $Operator.arms(x));$

        -- when the operator is touching the arm of the robot, the speed of the
        -- arm of the robot is none
        stoppedWhileContact:
            $Arm.contact() \implies $
                $RobotStatus.currentEndEffectorSpeed == None;$

        -- TODO
        -- TODO C'è un exists di troppo
        -- TODO
                
        -- if the operator is moving the end effector, then the end effector 
        -- will move even if the speed is none
        armSpeedNoneWithContact: 
            $\forall x(Status.currentArmSpeed == None \land$
              $Arm.contact() \land$
              $Arm.endEffector(x)\implies$
                  $\exists \future(\exists y Arm.endEffector(y) \land 
                    (x == y \lor Grid.adjacent(x, y)), 1));$
        
end RobotPositionSensor.
\end{lstlisting}
\pagebreak
\begin{lstlisting}[fontadjust, mathescape, frame=single] 
-- In this class we define all the predicates relative to the robot's bin 
-- and its velocity.
-- We defined an ordering on how the bin is filled and empty. We defined as 
-- capacity of our bin only three pieces.
class RobotStatus

    temporal domain integer;

    visible targetCartSpeed, targetEndEffectorSpeed, 
             binEmpty, binFull, addPieceToBin, removePieceFromBin,
             currentCartSpeed, currentEndEffectorSpeed;

    TD items:
        predicates
            binEmpty(),
            binFull(),
            addPieceToBin(),
            removePieceFromBin();

        vars
            targetCartSpeed({None, Low, Normal}),
            currentCartSpeed({None, Low, Normal}),
            targetEndEffectorSpeed({None, Low, Normal}),
            currentEndEffectorSpeed({None, Low, Normal}),
            binStatus({Empty, 1, 2, Full});

    axioms
        -- bin empty
        binIsEmpty:
            $binEmpty() \iff binStatus == Empty;$

        -- bin full
        binIsFull:
            $binFull() \iff binStatus == Full;$

        -- ordering of bin status
        ordering1:
            $binStatus == Empty \land addPieceToBin() \implies \exists x (\becomes(binStatus == 1, x));$
        ordering2:
            $binStatus == 1 \land addPieceToBin() \implies \exists x (\becomes(binStatus == 2, x));$
        ordering3:
            $binStatus == 2 \land addPieceToBin() \implies \exists x (\becomes(binStatus == Full, x));$
        ordering4:
            $binStatus == Full \land removePieceFromBin() \implies \exists x (\becomes(binStatus == 2, x));$
        ordering5:
            $binStatus == 2 \land removePieceFromBin() \implies \exists x (\becomes(binStatus == 1, x));$

        ordering6:
            $binStatus == 1 \land removePieceFromBin() \implies \exists x (\becomes(binStatus == Empty, x));$

        -- a piece can be added to the local bin only if there is 
        -- enough capacity
        stillNotFull:
            $addPieceToBin() \implies \neg binStatus == Full;$

        -- a piece can be removed from the local bin only if there is
        -- at least one in the bin
        stillNotEmpty:
            $removePieceFromBin() \implies \neg binStatus == Empty;$

        -- addPieceToBin() and removePieceFromBin() are instantaneous events
        singleInstantEvents1:
            $addPieceToBin() \implies \future(\neg addPieceToBin(), 1)$;
        singleInstantEvents2:
            $removePieceFromBin() \implies \future(\neg removePieceFromBin(), 1)$;

        -- binStatus stays constant if there is no action performed
        constantBin: 
            $\forall x (binStatus == x \land \neg (addPieceToBin() \lor removePieceFromBin()) \implies$
                $\future(binStatus == x, 1));$

end RobotStatus.
\end{lstlisting}

\pagebreak
\subsection{Definition of the RobotController class}
\begin{lstlisting}[fontadjust, mathescape, frame=single]
class RobotController
  -- RobotController handles all the actions that the robot can execute, 
  -- specifying preconditions, axioms over the duration of the action
  -- and the sequences of events that must happen during the action

  temporal domain integer;

  TD items

  predicates
    do({
    PickFromBinArea,
    DropToLocalBin,
    GoToTombstone,
    PickFromLocalBin,
    DropToTombstone,
    PickFromTombstone,
    GoToConveyorBelt,
    DropToConveyorBelt,
    GoToBinArea
    });

    isArmMoving(),
    PickFromBinAreaT0(), PickFromBinAreaT1(),
    PickFromBinAreaT2(), PickFromBinAreaT3(),
    DropToTombstoneT0(), DropToTombstoneT1(),
    DropToTombstoneT2(), DropToTombstoneT3(),
    PickFromTombstoneT0(), PickFromTombstoneT1(),
    PickFromTombstoneT2(), PickFromTombstoneT3();
    DropToConveyorBeltT0(), DropToConveyorBeltT1(),
    DropToConveyorBeltT2(), DropToConveyorBeltT3();

  variables
    currentAction({
    PickFromBinArea,
    DropToLocalBin,
    GoToTombstone,
    PickFromLocalBin,
    DropToTombstone,
    PickFromTombstone,
    GoToConveyorBelt,
    DropToConveyorBelt,
    GoToBinArea
    }),

    -- None -> Proceed normally with the execution
    -- Emergency -> Stop the robot and wait for the HDICommand to return to None
    -- Continue -> when the robot has placed a piece on the tombstone, it needs 
    -- to wait for the lavoration to be done and for the operator to signal that 
    -- the lavoration has terminated.

    HDICommand({
    None,
    Emergency,
    Continue
    });
    
  TI items 
  

  modules RobotStatus: RobotStatus,
                Robot: RobotPositionSensor;

  
  
  axioms

  -- CORRECT FLOW OF ACTIONS

    -- the actions must be done in the correct order
    correctActionOrder:
      $(currentAction == PickFromBinArea \limply $
        $\until(currentAction == PickFromBinArea, currentAction == DropToLocalBin))
      \ \land$

      -- After having dropped a piece in the local bin, the robot either takes 
      -- another or goes to the tombstone
      $(currentAction == DropToLocalBin \limply $
        $(\until(currentAction == DropToLocalBin,currentAction == PickFromBinArea)\ ||$
        $\until(currentAction == DropToLocalBin, currentAction == GoToTombstone)))
      \ \land$

      -- After having gone to the tombstone, the robot takes the stored working 
      -- piece with the end effector
      $(currentAction == GoToTombstone \limply $
        $(\until(currentAction == GoToTombstone, currentAction == PickFromLocalBin)))
      \ \land$

      -- After having picked up a piece from the local bin, the robot drops it 
      -- into the tombstone
      $(currentAction == PickFromLocalBin \limply $
        $\until(currentAction == PickFromLocalBin, currentAction == DropToTombstone))
      \ \land$

      -- After having dropped a piece into the tombstone, the robot waits for  
      -- the operator's signal and then picks up the reshaped piece
      $(currentAction == DropToTombstone \limply $ 
        $\until(currentAction == DropToTombstone, currentAction = PickFromTombstone))
      \ \land$

      -- After having picked up the piece from the tombstone, the robot goes to
      -- the conveyor belt
      $(currentAction == PickFromTombstone \limply $
        $\until(currentAction == PickFromTombstone, currentAction = GoToConveyorBelt))
      \ \land$


      -- After having gone to the conveyor belt, the robot drops the piece onto
      -- it
      $(currentAction == GoToConveyorBelt \limply $
        $\until(currentAction == GoToConveyorBelt, currentAction == DropToConveyorBelt)
      \ \land$

      -- After having dropped the piece onto the conveyor belt, the robot either 
      -- goes back to the tombstone or it goes to the bin area if local bin is 
      -- empty
      $(currentAction == DropToConveyorBelt \limply $
        $\until(currentAction == DropToConveyorBelt, currentAction == GoToTombstone)\ ||$
        $\until(currentAction == DropToConveyorBelt, currentAction == GoToBinArea))
      \ \land$

      -- After having gone to the bin area, the robot picks up a piece from the 
      -- bin area
      $(currentAction == GoToBinArea \limply $
        $\until(currentAction == GoToBinArea, currentAction == PickFromBinArea);$

    -- the change in currentAction must be preceded by a do request
    correctInit:
      $\forall x (\becomes(currentAction == x) \iff do(x));$

    
    -- during emergency mode, the robot needs to be stopped immediately
    emergencyMode:
      $(HDICommand == Emergency) \limply $
        $(RobotStatus.currentCartSpeed == None) \land (RobotStatus.currentEndEffectorSpeed == None);$

    -- do is an instantaneous event (lasts only one time instant)
    instantaneousDo:
        $\forall x (do(x) \limply \future(\neg do(x), 1));$

  -- DEFINITION OF PRECONDITIONS, AXIOMS HOLDING DURING THE ACTION 
  -- AND SEQUENCES OF EVENTS

  -- PickFromBinArea
    -- the robot has space in the local bin and is at the bin area
    prePickFromBinArea:
        $\becomes(currentAction == PickFromBinArea) \limply$
            $\neg RobotStatus.binFull() \land \neg Robot.Arm.holding() \land$
            $\forall x (Robot.Cart.position(x) \limply Grid.adjToBinArea(x));$

    -- the robot must be still
    duringPickFromBinArea:
        $currentAction == PickFromBinArea \limply RobotStatus.currentCartSpeed == None;$

    
    -- t0: Start -> end effector on top of cart, not holding
    -- t1: end effector on top of bin area, not holding
    -- t2: end effector on top of bin area, holding
    -- t3: End -> end effector on top of cart, holding
    $PickFromBinAreaT0() \iff (currentAction == PickFromBinArea \land \neg Robot.Arm.holding() \land$
        $\exists x (Robot.Cart.position(x) \land Robot.Arm.endEffector(x)))$;
    
    $PickFromBinAreaT1() \iff (currentAction == PickFromBinArea \land \neg Robot.Arm.holding() \land$
        $Robot.Arm.endEffector(LBA)))$;
    
    $PickFromBinAreaT2() \iff (currentAction == PickFromBinArea \land Robot.Arm.holding() \land$
        $Robot.Arm.endEffector(LBA)))$;
    
    $PickFromBinAreaT3() \iff (currentAction == PickFromBinArea \land Robot.Arm.holding() \land$
        $\exists x (Robot.Cart.position(x) \land Robot.Arm.endEffector(x)))$;
    
    -- to complete the PickFromBinArea action we need to have reached
    -- PickFromBinAreaT3()
    $\becomes(\neg currentAction == PickFromBinArea) \implies Past(PickFromBinAreaT3(), 1)$;
    
    -- from T3 to T2 the only thing that changes is the position
    -- of the end effector
    $PickFromBinAreaT3() \implies \exists t (\lasttime(PickFromBinAreaT2(), t) \land$
        $\lastedOp_{ie}(currentAction == PickFromBinArea \land Robot.Arm.holding(), t))$;

    -- from T2 to T1 the only thing that changes is the fact that
    -- the end effector is holding a piece
    $PickFromBinAreaT2() \implies \exists t (\lasttime(PickFromBinAreaT1(), t) \land$
        $\lastedOp_{ie}(currentAction == PickFromBinArea \land Robot.Arm.endEffector(LBA)))$;
    
    -- from T1 to T0 the only thing that changes is the position
    -- of the end effector
    $PickFromBinAreaT1() \implies \exists t (\lasttime(PickFromBinAreaT0(), t) \land$
        $\lastedOp_{ie}(currentAction == PickFromBinArea \land \neg Robot.Arm.holding()))$;
      
  -- DropToLocalBin
    -- the robot holds a piece, the local bin isn't full and the endEffector 
    -- is on the cart
    preDropToLocalBin:
        $\becomes(currentAction == DropToLocalBin) \limply $
            $\neg RobotStatus.binFull() \land Robot.Arm.holding() \land $
            $\exists x (Robot.Cart.position(x) \land \textit{Robot.Arm.endEffector}(x));$
    
    -- the robot cannot move, the endEffector must stay in the same zone
    duringDropToLocalBin:
        $currentAction == DropToLocalBin \limply $
            $(RobotStatus.currentCartSpeed == None \land$
            $RobotStatus.currentEndEffectorSpeed == None);$
  
    -- there exists a time t in which the robot switches from 
    -- holding to not holding
    dropSequence:
        $\becomes(\neg currentAction == DropToLocalBin) \limply $
            $\exists t (\lasttime(currentAction == DropToLocalBin \land Robot.Arm.holding(), t) \land$
            $\lastedOp_{ie}(currentAction == DropToLocalBin, t))$;

    -- after the drop to local bin action has finished, then the amount of pieces
    -- that are stored inside the local bin has increased by one unit
    -- $t_1$: the time when DropToLocalBin has started
    -- $t_2$: the time during which the piece has been dropped from the
    --     end effector to the local bin
    dropOnePieceToLocalBin:
        $\exists t_1 (\lasttime(\neg currentAction == DropToLocalBin, t_1) \land$
            $\exists t_2 (0 < t_2 < t_1 \land \past(RobotStatus.addPieceToBin(), t_2) \land$
                $\forall t_3 (0 < t_3 < t_1 \land t_3 \neq t_2 \implies \neg \past(RobotStatus.addPieceToBin(), t_3))))$;

  -- GoToTombstone
    -- the robot's local bin is full, it isn't holding any piece and the 
    -- endEffector is on the cart
    preGoToTombstone:
        $\becomes(currentAction == GoToTombstone) \limply $
            $RobotStatus.binFull() \land \neg RobotStatus.holding();$
    
    -- in case the robot stops himself during the action of going towards the 
    -- tombstone, then the endEffector cannot move 
    -- in RobotPositionSensor we  already stated that if the cart is moving then 
    -- the endEffector must stay on top of the cart
    duringGoToTombstone:
        $currentAction == GoToTombstone \limply $
            $(RobotStatus.currentCartSpeed == None \limply$ 
                $RobotStatus.currentEndEffectorSpeed == None);$



    -- there exists a time t smaller than MaxTravelTime when the robot is in a 
    -- position adjacent to the tombstone
    goToTombstoneSequence:
    $\becomes(\neg currentAction == GoToTombstone) \implies$
        $\exists x (Robot.Cart.position(x) \land Grid.adjToTombstone(x))$;

  -- PickFromLocalBin
    -- the robot isn't holding any piece, the local bin isn't empty and the 
    -- endEffector is in the same position as the cart
    prePickFromLocalBin:
        $\becomes(currentAction == PickFromLocalBin) \limply $
            $\neg RobotStatus.binEmpty() \land \neg Robot.Arm.holding()\ \land $
            $\exists x (Robot.Cart.position(x) \land \textit{Robot.Arm.endEffector}(x));$
    
    -- the robot cannot move if the action is PickFromLocalBin, and also the 
    -- endEffector does not have to move
    duringPickFromLocalBin:
        $currentAction == PickFromLocalBin \limply $
            $(RobotStatus.currentCartSpeed == None\ \land$
            $RobotStatus.currentEndEffectorSpeed == None);$
  
    postPickFromLocalBin:
        $\becomes(\neg currentAction == PickFromLocalBin) \implies Robot.Arm.holding() \land$
            $\exists x (Robot.Cart.position(x) \land Robot.Arm.endEffector(x))$;
            
    -- $t_1$ is the starting time of the action
    -- $t_2$ is the time during which the piece is taken from the local bin
    removedOnlyOnePieceFromBin:
        $\becomes(\neg currentAction == PickFromLocalBin) \implies$
            $\exists t_1 (\lasttime(\neg currentAction == PickFromLocalBin, t_1) \land$
                 $\exists t_2 (0 < t_2 < t_1 \land \past(RobotStatus.removePieceFromBin(), t_2) \land$
                      $\forall t_3 (0 < t_3 < t_1 \land t_2 \neq t_3 \implies \neg \past(RobotStatus.removePieceFromBin(), t_3)))$;
                      
  -- DropToTombstone
    -- the robot is holding a piece and it is in a position adjacent 
    -- to the tombstone
    preDropToTombstone:
        $\becomes(currentAction == DropToTombstone) \limply$
            $Robot.Arm.holding() \land \forall x (Robot.Cart.position(x) \limply Grid.adjToTombstone(x));$

    -- the robot cart isn't moving
    duringDropToTombstone:
        $currentAction == DropToTombstone \limply $
            $RobotStatus.currentCartSpeed == None;$

    -- t0: Start -> end effector on top of cart, holding
    -- t1: end effector on top of tombstone, holding
    -- t2: end effector on top of tombstone, not holding
    -- t3: End -> end effector on top of cart, not holding
    $DropToTombstoneT0() \iff (currentAction == DropToTombstone \land Robot.Arm.holding() \land$
        $\exists x (Robot.Cart.position(x) \land Robot.Arm.endEffector(x)))$;
    
    $DropToTombstoneT1() \iff (currentAction == DropToTombstone \land Robot.Arm.holding() \land$
        $Robot.Arm.endEffector(L0)))$;
    
    $DropToTombstoneT2() \iff (currentAction == DropToTombstone \land \neg Robot.Arm.holding() \land$
        $Robot.Arm.endEffector(L0)))$;
    
    $DropToTombstoneT3() \iff (currentAction == DropToTombstone \land \neg Robot.Arm.holding() \land$
        $\exists x (Robot.Cart.position(x) \land Robot.Arm.endEffector(x)))$;
    
    -- to complete the DropToTombstone action we need to have reached
    -- DropToTombstoneT3()
    $\becomes(\neg currentAction == DropToTombstone) \implies Past(DropToTombstoneT3(), 1)$;
    
    -- from T3 to T2 the only thing that changes is the position
    -- of the end effector
    $DropToTombstoneT3() \implies \exists t (\lasttime(DropToTombstoneT2(), t) \land$
        $\lastedOp_{ie}(currentAction == DropToTombstone \land \neg Robot.Arm.holding(), t))$;

    -- from T2 to T1 the only thing that changes is the fact that
    -- the end effector is holding a piece
    $DropToTombstoneT2() \implies \exists t (\lasttime(DropToTombstoneT1(), t) \land$
        $\lastedOp_{ie}(currentAction == DropToTombstone \land Robot.Arm.endEffector(L0)))$;
    
    -- from T1 to T0 the only thing that changes is the position
    -- of the end effector
    $DropToTombstoneT1() \implies \exists t (\lasttime(DropToTombstoneT0(), t) \land$
        $\lastedOp_{ie}(currentAction == DropToTombstone \land Robot.Arm.holding()))$;

  -- PickFromTombstone
    -- the robot has received the Continue HDI signal, it isn't holding anything 
    -- and it is in a position adjacent to the tombstone
    prePickFromTombstone:
        $\becomes(currentAction == PickFromTombstone) \limply $
            $HDICommand == Continue\ \land $
            $\neg Robot.Arm.holding()\ \land $
            $\forall x (Robot.Cart.position(x) \limply Grid.adjToTombstone(x));$

    -- the robot cart isn't moving
    duringPickFromTombstone:
        $currentAction == PickFromTombstone \limply RobotStatus.currentCartSpeed == None;$
        

    -- t0: Start -> end effector on top of cart, not holding
    -- t1: end effector on top of tombstone, not holding
    -- t2: end effector on top of tombstone, holding
    -- t3: End -> end effector on top of cart, holding
    $PickFromTombstoneT0() \iff (currentAction == PickFromTombstone \land \neg Robot.Arm.holding() \land$
        $\exists x (Robot.Cart.position(x) \land Robot.Arm.endEffector(x)))$;
    
    $PickFromTombstoneT1() \iff (currentAction == PickFromTombstone \land \neg Robot.Arm.holding() \land$
        $Robot.Arm.endEffector(L0)))$;
    
    $PickFromTombstoneT2() \iff (currentAction == PickFromTombstone \land Robot.Arm.holding() \land$
        $Robot.Arm.endEffector(L0)))$;
    
    $PickFromTombstoneT3() \iff (currentAction == PickFromTombstone \land Robot.Arm.holding() \land$
        $\exists x (Robot.Cart.position(x) \land Robot.Arm.endEffector(x)))$;
    
    -- to complete the PickFromTombstone action we need to have reached
    -- PickFromTombstoneT3()
    $\becomes(\neg currentAction == PickFromTombstone) \implies Past(PickFromTombstoneT3(), 1)$;
    
    -- from T3 to T2 the only thing that changes is the position
    -- of the end effector
    $PickFromTombstoneT3() \implies \exists t (\lasttime(PickFromTombstoneT2(), t) \land$
        $\lastedOp_{ie}(currentAction == PickFromTombstone \land Robot.Arm.holding(), t))$;

    -- from T2 to T1 the only thing that changes is the fact that
    -- the end effector is holding a piece
    $PickFromTombstoneT2() \implies \exists t (\lasttime(PickFromTombstoneT1(), t) \land$
        $\lastedOp_{ie}(currentAction == PickFromTombstone \land Robot.Arm.endEffector(L0)))$;
    
    -- from T1 to T0 the only thing that changes is the position
    -- of the end effector
    $PickFromTombstoneT1() \implies \exists t (\lasttime(PickFromTombstoneT0(), t) \land$
        $\lastedOp_{ie}(currentAction == PickFromTombstone \land \neg Robot.Arm.holding()))$;

  -- GoToConveyorBelt
    -- the robot is holding a piece
    preGoToConveyorBelt:
        $\becomes(currentAction == GoToConveyorBelt) \limply RobotStatus.holding();$
    
    -- in case the robot stops himself during the action of going towards the 
    -- conveyor belt, then the endEffector cannot move 
    -- in RobotPositionSensor we already stated that if the cart is moving then 
    -- the endEffector must stay on top of the cart
    duringGoToConveyorBelt:
        $currentAction == GoToConveyorBelt \limply $
            $((RobotStatus.currentCartSpeed == None$ 
                $\limply RobotStatus.currentEndEffectorSpeed == None )$
            $\land Robot.Arm.holding());$

    postGoToConveyorBelt:
        $\becomes(\neg currentAction == GoToConveyorBelt) \implies$
             $\exists x (Robot.Cart.position(x) \land Grid.adjToConveyorBelt(x))$;
      
  -- DropToConveyorBelt
    -- the robot is holding a piece and it is in a position adjacent to the 
    -- conveyor belt
    preDropToConveyorBelt:
      $\becomes(currentAction == DropToConveyorBelt) \limply $
        $Robot.Arm.holding() \land \forall x (Robot.Cart.position(x) \limply $
          $Grid.adjToConveyorBelt(x));$
    
    -- the robot cart isn't moving
    duringDropToConveyorBelt:
      $currentAction == DropToConveyorBelt \limply RobotStatus.currentCartSpeed == None;$

    -- t0: Start -> end effector on top of cart, holding
    -- t1: end effector on top of conveyor belt, holding
    -- t2: end effector on top of conveyor belt, not holding
    -- t3: End -> end effector on top of cart, not holding
    $DropToConveyorBeltT0() \iff (currentAction == DropToConveyorBelt \land Robot.Arm.holding() \land$
        $\exists x (Robot.Cart.position(x) \land Robot.Arm.endEffector(x)))$;
    
    $DropToConveyorBeltT1() \iff (currentAction == DropToConveyorBelt \land Robot.Arm.holding() \land$
        $Robot.Arm.endEffector(LCB)))$;
    
    $DropToConveyorBeltT2() \iff (currentAction == DropToConveyorBelt \land \neg Robot.Arm.holding() \land$
        $Robot.Arm.endEffector(LCB)))$;
    
    $DropToConveyorBeltT3() \iff (currentAction == DropToConveyorBelt \land \neg Robot.Arm.holding() \land$
        $\exists x (Robot.Cart.position(x) \land Robot.Arm.endEffector(x)))$;
    
    -- to complete the DropToConveyorBelt action we need to have reached
    -- DropToConveyorBeltT3()
    $\becomes(\neg currentAction == DropToConveyorBelt) \implies Past(DropToConveyorBeltT3(), 1)$;
    
    -- from T3 to T2 the only thing that changes is the position
    -- of the end effector
    $DropToConveyorBeltT3() \implies \exists t (\lasttime(DropToConveyorBeltT2(), t) \land$
        $\lastedOp_{ie}(currentAction == DropToConveyorBelt \land \neg Robot.Arm.holding(), t))$;

    -- from T2 to T1 the only thing that changes is the fact that
    -- the end effector is holding a piece
    $DropToConveyorBeltT2() \implies \exists t (\lasttime(DropToConveyorBeltT1(), t) \land$
        $\lastedOp_{ie}(currentAction == DropToConveyorBelt \land Robot.Arm.endEffector(LCB)))$;
    
    -- from T1 to T0 the only thing that changes is the position
    -- of the end effector
    $DropToConveyorBeltT1() \implies \exists t (\lasttime(DropToConveyorBeltT0(), t) \land$
        $\lastedOp_{ie}(currentAction == DropToConveyorBelt \land Robot.Arm.holding()))$;

  -- GoToBinArea
    -- the robot doesn't hold any piece and the bin is empty
    preGoToBinArea:
        $\becomes(currentAction == GoToBinArea) \limply $
            $\neg RobotStatus.holding() \land RobotStatus.binEmpty();$
    
    -- in case the robot stops himself during the action of going towards the 
    -- bin area, then the endEffector cannot move
    -- in RobotPositionSensor we already stated that if the cart is moving then 
    -- the endEffector must stay on top of the cart
    duringGoToBinArea:
        $currentAction == GoToBinArea \limply $
            $(RobotStatus.currentCartSpeed == None \limply$
                $RobotStatus.currentEndEffectorSpeed == None);$

    postGoToBinArea:
        $\becomes(\neg currentAction == GoToBinArea) \limply $
              $\exists x (Robot.Cart.position(x)\ \land Grid.adjToBinArea(x))$;
                      
    -- The robot can move the arm only when the cart is not moving
    isArmMoving: 
        $isArmMoving() \iff$
            $\neg (currentAction == GoToBinArea \lor currentAction == GoToTombstone \lor$
            $currentAction == GoToConveyorBelt);$
         
end RobotController.
\end{lstlisting}

%%% Local Variables:
%%% mode: latex
%%% TeX-master: "../document"
%%% End:


\pagebreak
\subsection{Definition of the PositionController class}
\begin{lstlisting}[fontadjust, mathescape, frame=single] 
class PositionController
    temporal domain integer;

    modules  Operator: OperatorPositionSensor,
             Robot: RobotPositionSensor,
             RobotStatus: RobotStatus,
             RobotController: RobotController,
             Grid: Grid;
    
        axioms
    
        -----------------------
        -- SAFETY ARM AXIOMS --
        -----------------------
        
        -- if the end effector is in the same area as the head of the operator, then the end effector is stopped
        headSameAreaAsRobotArm:$
            \forall x (Robot.Arm.position(x) \land Operator.head(x) \land
                <azioni in cui può muovere il braccio> \limply
                RobotStatus.targetEndEffectorSpeed == None);$
        -- if the head of the operator is close to the arm of the robot, then the arm is either moving slowly or staying still
        headCloseToRobotArm:$
            \forall x, y (x != y \land Robot.Arm.position(x) \land Operator.head(y) \land Grid.adjacent(x, y) \land
                <azioni in cui può muovere il braccio> \limply
                (RobotStatus.targetEndEffectorSpeed == None || RobotStatus.targetEndEffectorSpeed == Low));$

        -- if the arms of the operator are in the same area as the arm
        -- of the robot, then either the arm of the robot is staying
        -- still or is moving slow.
        sameAreaArms:$
            \forall x (Robot.Arm.position(x) \land Operator.arms(x) \land \neg Robot.Arm.contact() \limply
                 RobotStatus.currentEndEffectorSpeed == None || RobotStatus.currentEndEffectorSpeed == Low);$

        -- if the arms of the operator are touching the arm of the robot, then the arm is not moving
        sameAreaArmsWithContact:$
            \forall x (Robot.Arm.position(x) \land Operator.arms(x) \land Robot.Arm.contact() \limply
                 RobotStatus.currentEndEffectorSpeed == None);$

        -- if the arm of the robot is in an area adjacent to the arms
        -- of the operator, then the robot is at most moving slow
        armsCloseArm:$
            \forall x, y (Robot.Arm.position(x) \land Operator.arms(y) \land x != y \land Grid.adjacent(x, y) \limply
                 RobotStatus.currentEndEffectorSpeed == None || RobotStatus.currentEndEffectorSpeed == Slow);$

        -- if the arm of the robot is in the same area as the operator's body, then the arm is not moving
        armCloseBody:$
            \forall x (Robot.Arm.position(x) \land Operator.body(x) \limply
                 RobotStatus.currentEndEffectorSpeed == None);$

        -- if the arm of the robot is in an area adjacent to the body
        -- of the operator, then the robot is at most moving slow
        bodyCloseArm:$
            \forall x, y (Robot.Arm.position(x) \land Operator.body(y) \land x != y \land Grid.adjacent(x, y) \limply
                 RobotStatus.currentEndEffectorSpeed == None || RobotStatus.currentEndEffectorSpeed == Slow);$

        ------------------------       
        -- SAFETY CART AXIOMS --
        ------------------------

        -- if the operator is in an area close to the wall and the
        -- robot is in an area adjacent to it, then the robot will not
        -- move to an area adjacent to the one occupied by the
        -- operator
        operatorCloseToWall:$
            \forall x, y (x != y \land Operator.body(x) \land Grid.border(x) \land Robot.Cart.position(y) \land Grid.adjacent(x, y) \limply
                \exists z (\  future((RobotStatus.currentCartSpeed == Low || RobutStatus.currentCardSpeed == None) \land
                           Robot.Cart.position(z) \land (z == y || Grid.adjacent(z, y) \land \neg Grid.adjacent(z, x))), 1));$

        -- if the operator is in a dangerous area and the
        -- robot is in an area adjacent to it, then the robot will not
        -- move to an area adjacent to the one occupied by the
        -- operator
        operatorDangerZone:$
            \forall x, y (x != y \land Operator.body(x) \land Grid.danger(x) \land Robot.Cart.position(y) \land Grid.adjacent(x, y) \limply
                \exists z (\  future((RobotStatus.currentCartSpeed == Low || RobutStatus.currentCardSpeed == None) \land
                           Robot.Cart.position(z) \land (z == y || Grid.adjacent(z, y) \land \neg Grid.adjacent(z, x))), 1));$


        -- if the operator is in a transit area and the robot is in
        -- the same area, then the robot needs to move slow
        operatorTransitZone:$
            \forall x (Operator.body(x) \land Grid.transit(x) \land Robot.Cart.position(x) \limply
                RobotStatus.currentCartSpeed == Low || RobotStatus.currentCartSpeed == None);$

end PositionController                
\end{lstlisting}


\pagebreak
%\subsection{Definition of the robot position sensors}
\section{Demonstration of Saftety Properties}
\subsection{No head contact}

\begin{lstlisting}[fontadjust, mathescape, frame=single]

-- if the end effector is in the same area as the head of the operator, 
-- then the end effector is stopped
  headSameAreaAsRobotArm:
    $\forall x (Robot.Arm.position(x) && Operator.head(x)\ && $
      $RobotController.isArmMoving() \implies$
        $RobotStatus.targetEndEffectorSpeed == None);$

\end{lstlisting}



\pagebreak
\appendix
\section{Area modeling}

\begin{align*}
&\neg Adj(L0, L0) \land Adj(L0, L1) \land Adj(L0, L2) \land Adj(L0, L3) \land Adj(L0, L4) \land \neg Adj(L0, L5) \land \neg Adj(L0, L6) \\
&\land \neg Adj(L0, L7) \land \neg Adj(L0, L8) \land \neg Adj(L0, L9) \land \neg Adj(L0, L10) \land \neg Adj(L0, L11) \land \neg Adj(L0, L12) \\
&\land \neg Adj(L0, L13) \land \neg Adj(L0, L14) \land \neg Adj(L0, L15) \land \neg Adj(L0, L16) \land \neg Adj(L0, L17) \land \neg Adj(L0, L18) \\
&\land \neg Adj(L0, L19) \land \neg Adj(L0, L20) \land \neg Adj(L0, L21) \land \neg Adj(L0, L22) \land \neg Adj(L0, L23) \land \neg Adj(L0, L24) \\
&\land \neg Adj(L0, L25) \land \neg Adj(L0, L26) \land \neg Adj(L0, L27) \land \neg Adj(L0, L28) \land Adj(L1, L0) \land \neg Adj(L1, L1) \land Adj(L1, L2) \\
&\land \neg Adj(L1, L3) \land \neg Adj(L1, L4) \land Adj(L1, L5) \land Adj(L1, L6) \land \neg Adj(L1, L7) \land \neg Adj(L1, L8) \land \neg Adj(L1, L9) \\
&\land \neg Adj(L1, L10) \land \neg Adj(L1, L11) \land \neg Adj(L1, L12) \land \neg Adj(L1, L13) \land \neg Adj(L1, L14) \land \neg Adj(L1, L15) \\
&\land \neg Adj(L1, L16) \land \neg Adj(L1, L17) \land \neg Adj(L1, L18) \land \neg Adj(L1, L19) \land \neg Adj(L1, L20) \land \neg Adj(L1, L21) \\
&\land \neg Adj(L1, L22) \land \neg Adj(L1, L23) \land \neg Adj(L1, L24) \land \neg Adj(L1, L25) \land \neg Adj(L1, L26) \land \neg Adj(L1, L27) \\
&\land \neg Adj(L1, L28) \land Adj(L2, L0) \land Adj(L2, L1) \land \neg Adj(L2, L2) \land Adj(L2, L3) \land \neg Adj(L2, L4) \land Adj(L2, L5) \\
&\land Adj(L2, L6) \land Adj(L2, L7) \land \neg Adj(L2, L8) \land \neg Adj(L2, L9) \land \neg Adj(L2, L10) \land \neg Adj(L2, L11) \\
&\land \neg Adj(L2, L12) \land \neg Adj(L2, L13) \land \neg Adj(L2, L14) \land \neg Adj(L2, L15) \land \neg Adj(L2, L16) \land \neg Adj(L2, L17) \\
&\land \neg Adj(L2, L18) \land \neg Adj(L2, L19) \land \neg Adj(L2, L20) \land \neg Adj(L2, L21) \land \neg Adj(L2, L22) \land \neg Adj(L2, L23) \\
&\land \neg Adj(L2, L24) \land \neg Adj(L2, L25) \land \neg Adj(L2, L26) \land \neg Adj(L2, L27) \land \neg Adj(L2, L28) \land Adj(L3, L0) \\
&\land \neg Adj(L3, L1) \land Adj(L3, L2) \land \neg Adj(L3, L3) \land Adj(L3, L4) \land \neg Adj(L3, L5) \land Adj(L3, L6) \\
&\ldots 
\end{align*}
\end{document}